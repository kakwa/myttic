\documentclass[11pt,twocolumn]{article}
\usepackage[utf8]{inputenc}

%opening
\title{Rapport Simres}
\author{Carpentier Pierre-François - Cissé Nouhoum } 

\begin{document}

\maketitle


\section{Simulation}
Dans un réseau, il y a plusieurs classes de flux qui circulent, ces flux n'ont pas tous les mêmes caractéristiques.
Par exemple les flux de données sont proches d'un comportement pouvant étre décrit par une loi de Poisson,
 les flux de voix peuvent être décrits par une loi exponentielle, et les flux vidéo émettent en continue.
De plus ils n'impliquent pas non plus les mêmes contraintes, que ca soit en terme de delais, de perte de paquets ou de bande passante.
Par exemple la voix nécessite une bande passante faible, mais aussi un delais faible (afin de maintenir une qualite dans la conversation), les flux videos et de données
nécessite une bande passante importante.

Nous simulons ici un lien 100Mb/s, sur ce lien circule trois trafics, un trafic voix (20Mb/s, taille de paquet 800 bits, décrit par une loi exponentielle),
 un trafic de vidéo (30Mb/s, taille de paquet 8000 bits), 
et un trafic de données (30Mb/s, de plus le trafic de données comporte plusieurs tailles de paquets (40\% de taille 400, 30\% de taille 4000, 
30\% de taille 12000, le trafic est décrit par une loi de Poisson).
Nous allons ici mettre en place de la qualité de service afin de différencier les trois flux et analyser le comportement de ces trois flux dans divers 
situation.

La simulation est réalisé en ns-2

\section{Résultats et analyse}

\end{document}
